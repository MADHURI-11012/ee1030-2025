\let\negmedspace\undefined
\let\negthickspace\undefined
\documentclass[journal]{IEEEtran}
\usepackage[a5paper, margin=10mm, onecolumn]{geometry}
%\usepackage{lmodern} % Ensure lmodern is loaded for pdflatex
\usepackage{tfrupee} % Include tfrupee package

\setlength{\headheight}{1cm} % Set the height of the header box
\setlength{\headsep}{0mm}     % Set the distance between the header box and the top of the text

\usepackage{gvv-book}
\usepackage{gvv}
\usepackage{cite}
\usepackage{amsmath,amssymb,amsfonts,amsthm}
\usepackage{algorithmic}
\usepackage{graphicx}
\usepackage{textcomp}
\usepackage{xcolor}
\usepackage{txfonts}
\usepackage{listings}
\usepackage{enumitem}
\usepackage{mathtools}
\usepackage{gensymb}
\usepackage{comment}
\usepackage[breaklinks=true]{hyperref}
\usepackage{tkz-euclide} 
\usepackage{listings}
% \usepackage{gvv}                                        
\def\inputGnumericTable{}                                 
\usepackage[latin1]{inputenc}                                
\usepackage{color}                                            
\usepackage{array}                                            
\usepackage{longtable}                                       
\usepackage{calc}                                             
\usepackage{multirow}                                         
\usepackage{hhline}                                           
\usepackage{ifthen}                                           
\usepackage{lscape}
\usepackage{multicol}
\begin{document}

\bibliographystyle{IEEEtran}
\vspace{3cm}

\title{12.338}
\author{EE25BTECH11012-BEERAM MADHURI}
% \maketitle
% \newpage
% \bigskip
{\let\newpage\relax\maketitle}

\renewcommand{\thefigure}{\theenumi}
\renewcommand{\thetable}{\theenumi}
\setlength{\intextsep}{10pt} % Space between text and floats


\numberwithin{equation}{enumi}
\numberwithin{figure}{enumi}
\renewcommand{\thetable}{\theenumi}


\textbf{Question}:\\
For a real symmetric matrix $\mathbf{A}$, which of the following statements is true?

\begin{enumerate}
    \item[a)] The matrix is always diagonalizable and invertible.
    \item[b)] The matrix is always invertible but not necessarily diagonalizable.
    \item[c)] The matrix is always diagonalizable but not necessarily invertible.
    \item[d)] The matrix is always neither diagonalizable nor invertible.
\end{enumerate}

\textbf{Solution:}\\
Checking for diagonalizability of matrix $A$ \\
given,
\begin{align}
\vec{A} = \vec{A}^\top
\end{align}
$\therefore$ eigenvalues of $\vec{A}$ are real.\\
for distinct eigenvalues $\lambda_i$, $\lambda_j$ corresponding eigenvectors are $\vec{x_i}$, $\vec{x_j}$.
\begin{align}
\vec{Ax_i} &= \lambda_i \vec{x_i} \quad \text{and} \quad \vec{Ax_j} = \lambda_j \vec{x_j} 
\\\vec{x_j}^\top \vec{A x_i} &= \lambda_i \vec{x_j}^\top \vec{x_i} \\
\vec{(Ax_j)}^\top \vec{x_i} &= \lambda_i \vec{x_j}^\top \vec{x_i}\\
\because \quad \vec{Ax_j} &= \lambda_j \vec{x_j}\\
\lambda_j \vec{x_j}^\top \vec{x_i} = \lambda_i \vec{x_j}^\top \vec{x_i}\\
(\lambda_j - \lambda_i) \vec{x_j}^\top \vec{x_i} = 0
\end{align}
\begin{align}
\lambda_i \neq \lambda_j \\
\vec{x_j}^\top\vec{x_i}=0
\end{align}
$\therefore$ \text{eigenvectors are orthogonal}\\
$\therefore$ \text{We can construct an orthogonal matrix with these eigenvectors}\\
\begin{align}
Q = [\vec{x_1} \ \vec{x_2} \ \vec{x_3} \ \dots \ \vec{x_n}] \\
Q^\top Q = I\\
A = Q MQ^\top
\end{align}
\text{Where $\vec{M}$ is diagonal matrix with it's entries as  eigen values of matrix A}\\
\text{The eigenvalues are placed in the same order as their corresponding eigenvectors $\mathbf{x}_1, \mathbf{x}_2, \ldots, \mathbf{x}_n$ }\\
\begin{align}
M = 
\begin{pmatrix}
\lambda_{1} & 0 & \cdots & 0 \\
0 & \lambda_{2} & \cdots & 0 \\
\vdots & \vdots & \ddots & \vdots \\
0 & 0 & \cdots & \lambda_{n}
\end{pmatrix}
\end{align}
where $\lambda_1$, $\lambda_2$, $\cdots$ $\lambda_n$ are eigen values of $\vec{A}$\\
$\therefore$ $\vec{A}$ is always diagonalizable.\\\\
Checking for invertibility of Matrix  $\vec{A}$:
\begin{align}
\vec{A} = Q M Q^\top \\|\vec{A}| = |Q| |M| |Q^\top|\\
|\vec{A}| = M_1 M_2 \cdots M_n
\end{align}
where $M_1, M_2, \cdots M_n$ are diagonal entries of Matrix M.\\
$\vec{A}$ is invertible only when 
\begin{align}
\det(\vec{A}) \neq 0 
\end{align}
that is $M_1, M_2, M_3 \cdots M_n \neq 0$ \\
that is none of its eigenvalues are zero \\

if $\lambda_i = 0$ \\
then $A$ is non-invertible \\

$\therefore$ a real symmetric matrix may or may not be invertible.\\
$\therefore$ Option c is correct.\\\\
Examples of a real symmetric matrix $\vec{A}$:\\
1.
\begin{align}
    \vec{A}=\begin{pmatrix}
        1 & 1\\1& 1
    \end{pmatrix}\\
    \vec{A}=\vec{A^\top}
\end{align}
$\vec{A}$ is real symmetric and diagonalizable but not invertible as $det(\vec{A})=0$\\
Eigen values of $\vec{A}$ are 0 and 2.
Eigen vectors of $\vec{A}$ are:
\begin{align}
    q1=\begin{pmatrix}
        1/\sqrt2\\-1/\sqrt2
    \end{pmatrix}\\
    q2=\begin{pmatrix}
        1/\sqrt2\\1/\sqrt2
    \end{pmatrix}
\end{align}
\begin{align}
    \vec{A}=\vec{QMQ^\top}\\
    where, \vec{Q}=\begin{pmatrix}
        1/\sqrt2 & 1/\sqrt2\\-1/\sqrt2 &1/\sqrt2
    \end{pmatrix}\\
    \vec{M}=\begin{pmatrix}
        0&0\\0&2
    \end{pmatrix}
\end{align}
2.
\begin{align}
    \vec{A}=\begin{pmatrix}
        2 & 1\\1& 2
    \end{pmatrix}\\
    \vec{A}=\vec{A^\top}
\end{align}
$\vec{A}$ is real symmetric, diagonizable and invertible matrix\\
Eigen values of $\vec{A}$ are 3 and 1.
Eigen vectors of $\vec{A}$ are:
\begin{align}
    q1=\begin{pmatrix}
        1/\sqrt2\\1/\sqrt2
    \end{pmatrix}\\
    q2=\begin{pmatrix}
        1/\sqrt2\\-1/\sqrt2
    \end{pmatrix}
\end{align}
\begin{align}
    \vec{A}=\vec{QMQ^\top}\\
    where, \vec{Q}=\begin{pmatrix}
        1/\sqrt2 & 1/\sqrt2\\1/\sqrt2 &-1/\sqrt2
    \end{pmatrix}\\
    \vec{M}=\begin{pmatrix}
        3&0\\0&1
    \end{pmatrix}
\end{align}
det$\vec{A}$$\neq$0
\begin{align}
    \vec{A}^-1=\begin{pmatrix}
        2/3 & -1/3\\-1/3 &2/3
    \end{pmatrix}
    \end{align}
\end{document}
